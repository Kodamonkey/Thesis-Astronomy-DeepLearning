\secnumbersection{CONCLUSIONES Y TRABAJO FUTURO}

\subsection{Resumen de aportes y respuestas a las preguntas}
Se sintetizan las contribuciones técnicas del pipeline DRAFTS-MB, su desempeño en banda L y su extensión a frecuencias milimétricas, respondiendo las preguntas de investigación sobre sensibilidad y precisión comparativas.

\subsection{Lecciones aprendidas de llevar DRAFTS a producción}
Se discuten hallazgos sobre integración de modelos, control del \textit{trials factor}, robustez a RFI y gestión de memoria/latencia en operación \textit{near-real-time}.

\subsection{Próximos pasos}
Se delinean reentrenos específicos para mm-wave, adaptación de dominio, procesamiento en flujo continuo (streaming) e integración con pipelines de observatorios.

\textbf{TODO}: Priorizar 3 acciones con plazos tentativos (p.ej., reentrenos mm-wave, pruebas en Band 3, integración con backend).

\subsection{Líneas de trabajo futuro}
Se proponen líneas para localización y consistencia por sub-bandas, validaciones cruzadas con pipelines externos (CHIME/FRB, MeerTRAP, ASKAP/CRAFT), y estrategias FAIR para datos/modelos. Si la detección falla en mm-wave por falta de \textit{bow-tie} (línea casi recta en tiempo--DM), reentrenar/adaptar el modelo o desarrollar uno nuevo específico para regímenes milimétricos con \textit{plots} característicos.
