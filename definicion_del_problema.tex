\secnumbersection{DEFINICIÓN DEL PROBLEMA}

% Se debe definir el problema, es importante no confundir definir el problema con describir la solución. Por ejemplo: ``diseñar una arquitectura e implementar una plataforma ...'' es una solución, no un problema.
% 
% Algunos elementos que podrían ir en este capítulo son (no es necesario que vayan todos):
% \begin{itemize}
%     \item Breve descripción del contexto donde se realizará la memoria (organización, línea dentro de la Informática en la que se basa, etc.)
%     \item ¿Qué y cómo se realiza actualmente la situación que mejorarás con tu memoria?
%     \item ¿Qué actores o usuarios están involucrados?
%     \item ¿Qué dificultades tienen esos actores actualmente? ¿cuántos son? (ideal si se pueden poner estadísticas para así saber si existe un mercado razonable para la solución que propondrás en tu memoria, en el fondo saber cuántas personas u organizaciones tienen el mismo problema que estás definiendo)
%     \item ¿Qué podría pasar si en el corto o mediano plazo no se solucionan esas dificultades (¿es decir, si no se hiciera tu memoria, qué pasaría?; en el fondo justificar por qué conviene hacer tu memoria, ¿cuál es la motivación o interés de hacerla?).
%     \item ¿Qué competencia existe actualmente? (a lo mejor ya existe una solución al problema, pero por qué no sirve, o por qué tu solución sería mejor, también se puede enfocar a si este problema existe en otras realidades y cómo ha sido solucionado allí).
%     \item Precisar los objetivos y alcances de la memoria (o solución al problema).
% \end{itemize}
% 
% En este capítulo, de ser necesario puede usar referencias bibliográficas (velar porque sean recientes), una cita de ejemplo \cite{schwab2002cure} y otras más \cite{georget1994study,beaumont1990patient}.
% 
% Recuerde poner notas al pie de página que sean explicativas \footnote{Este es un ejemplo de una nota al pie de página. Puede indicar alguna URL, definiciones, aclarar alguna información pertinente del texto, citar algunas referencias, etc..}.
% 
% \subsection{SUBSECCIÓN DE PRUEBA}
% 
% Sed ut perspiciatis unde omnis iste natus error sit voluptatem accusantium doloremque laudantium, totam rem aperiam, eaque ipsa quae ab illo inventore veritatis et quasi architecto beatae vitae dicta sunt explicabo.
% 
% \subsubsection{SUBSUBSECCIÓN DE PRUEBA}
% 
% Nemo enim ipsam voluptatem quia voluptas sit aspernatur aut odit aut fugit, sed quia consequuntur magni dolores eos qui ratione voluptatem sequi nesciunt. Neque porro quisquam est, qui dolorem ipsum quia dolor sit amet.
% 
% \subsubsection{OTRA SUBSUBSECCIÓN DE PRUEBA}
% 
% Nemo enim ipsam voluptatem quia voluptas sit aspernatur aut odit aut fugit, sed quia consequuntur magni dolores eos qui ratione voluptatem sequi nesciunt. Neque porro quisquam est, qui dolorem ipsum quia dolor sit amet.

Los \textit{Fast Radio Bursts} (FRBs) son fenómenos de milisegundos de origen extragaláctico. Su estudio se ha consolidado como una línea de trabajo en astronomía de radio y ciencia de datos, con impacto en instrumentación, procesamiento de señales y aprendizaje automático. Este capítulo formula el \textbf{problema} que aborda la memoria: la ausencia de un \textit{pipeline} operativo, reproducible y portable que permita \textbf{detectar y clasificar FRBs en tiempo (casi) real} y que, además, sea \textbf{extensible a regímenes de alta frecuencia} (mm-wave) sin reentrenar modelos. La investigación se sitúa en la intersección de la Informática (ingeniería de software y ML aplicado) y la radioastronomía, utilizando como base el marco conceptual y el código no finalizado de \textbf{DRAFTS}\footnote{\emph{Deep-learning Real-time pAstro nomy FRB TranSients} (DRAFTS). Se usa como punto de partida por disponer de modelos de detección y clasificación y de utilidades de procesamiento, pero su código requiere consolidación para uso productivo.}, sobre el cual se propone construir un \textit{pipeline} robusto.

\medskip
Hoy, la búsqueda de FRBs en bandas decimétricas se apoya en flujos clásicos (limpieza RFI, dedispersión en rejillas de DM, filtrado por \emph{matched filtering}, generación y depuración de candidatos). La depuración sigue demandando curaduría manual y scripts ad hoc por equipo. En \textbf{alta frecuencia} (30--100,GHz; modo \emph{phased} en arreglos como ALMA\footnote{\emph{Phased array} permite formar un haz coherente y registrar series temporales de alta resolución para experimentos de transientes/pulsos.}), la ventana científica es prometedora pero \textbf{no existen \textit{pipelines} estandarizados y de propósito general} para transientes de milisegundos: los datos son distintos (menor retardo dispersivo observable, otras sistemáticas) y los desarrollos actuales son prototipos.

\medskip

El problema afecta a: (i) astrónomos/as de radio que planifican, ejecutan y validan búsquedas; (ii) ingenieros/as de datos que mantienen \textit{pipelines} y orquestación; (iii) observatorios con backends de alta tasa (p.,ej., FAST, ASKAP, MeerKAT, GBT/VLA, ALMA en \emph{phased}). El número de grupos trabajando activamente en FRBs y transientes de milisegundo es amplio y distribuido en universidades y centros; muchos de ellos replican esfuerzos de ingeniería similares para cubrir sus necesidades locales. Un \textit{pipeline} generalista y portable reduce esa duplicación y mejora el retorno científico de tiempo de telescopio.

\medskip

Algunas de las dificultades observadas son:

\begin{enumerate}
\item \textbf{Tiempo de ejecución y fricción operativa:} archivos grandes, dedispersión en rejillas amplias y visualización masiva de candidatos encarecen I/O y memoria; la latencia dificulta alertas y seguimiento.
\item \textbf{Altas tasas de falsos positivos:} RFI y artefactos instrumentales exigen segundas cribas y validación experta.
\item \textbf{Brecha en alta frecuencia:} a mm-wave el retardo $\Delta t\propto \mathrm{DM},\nu^{-2}$ es menor; patrones diagnósticos (\emph{bow-tie} en tiempo--DM) se atenúan y los productos clásicos pierden poder discriminante. Falta un flujo adaptado a estas condiciones.
\item \textbf{Modelos sin producto:} existen trabajos con redes neuronales (detección y clasificación), pero \textbf{no} siempre hay \textit{pipelines} \emph{end-to-end} maduros y de propósito general que integren modelos, ingeniería de datos, métricas y reportabilidad.
\end{enumerate}

\medskip
Sin una solución, se mantiene la latencia (pérdida de oportunidades de \emph{follow-up}), se perpetúan sesgos por curaduría manual, se multiplica el costo de mantenimiento de soluciones ad hoc y se desaprovecha la ventana en mm-wave (donde podrían aparecer firmas diferentes a las conocidas). También se diluye la trazabilidad/reproducibilidad, limitando comparaciones entre proyectos.

\medskip
Hoy por hoy existen herramientas clásicas (PRESTO/Heimdall) y marcos específicos de proyectos que han sido eficaces en L/S-band, pero su adopción como \textbf{producto} general es limitada, pues no se consideran escenarios de alta frecuencia. En DL, \textbf{DRAFTS} sugiere una vía: detección en mapas tiempo--DM (estimación de $(t_{arr},,\mathrm{DM})$) + \emph{patch} tiempo--frecuencia para clasificación binaria. La brecha es llevar ese enfoque a un \textit{pipeline} reproducible, con ingeniería de software, monitoreo, métricas y \textbf{adaptación a alta frecuencia} sin reentrenar modelos.

\medskip
Entonces, luego del contexto anterior, presentamos la formulación del problema; dado un flujo de datos de radioastronomía (FITS/PSRFITS u otros formatos) y dos modelos preentrenados (detección y clasificación), se requiere construir un \textbf{pipeline operativo, reproducible y portable} que: (i) procese en lotes y en línea con control de recursos, (ii) reduzca falsos positivos con una segunda criba basada en DL, (iii) produzca salidas auditables (catálogo de candidatos, recortes, figuras, métricas), y (iv) \textbf{funcione en alta frecuencia} parametrizando adecuadamente DM-grids, escalas temporales y productos diagnósticos (incluida polarización cuando esté disponible). Todo ello \textbf{sin reentrenar} los modelos base.

\medskip
La propuesta ataca un cuello de botella real (latencia + robustez) y agrega valor transversal (portabilidad y estandarización). El impacto esperado es doble: (i) \emph{operativo}, al disminuir fricción y tiempos de análisis; (ii) \emph{científico}, al habilitar búsquedas consistentes en mm-wave y facilitar comparabilidad entre campañas.

\medskip
\noindent\textbf{Objetivos}
\begin{itemize}
\item \textbf{General:} Desarrollar un \textbf{pipeline astronómico} para \textbf{detección y clasificación de FRBs} basado en dos CNN preentrenadas, \textbf{extendido a alta frecuencia} sin reentrenamiento.
\item \textbf{Específicos:}
\begin{enumerate}
\item Integrar los modelos de \emph{detección} y \emph{clasificación} y construir alrededor un flujo robusto: ingesta $\to$ preprocesamiento $\to$ inferencia $\to$ posprocesamiento $\to$ reporte.
\item Aplicar ingeniería de software y algoritmos: manejo de I/O, \emph{chunking/overlap}, vectorización/aceleración, control de memoria y registro de trazas.
\item Adaptar el flujo a \textbf{alta frecuencia} por parametrización (DM-grid, ventanas temporales, productos en Stokes/polarización si existen), evitando reentrenamiento.
\item Validar sobre datos previamente analizados, \textbf{igualando o superando} el recuento de eventos reportados y midiendo latencia, \emph{precision} y \emph{recall}.
\item Realizar análisis exploratorios para identificar \textbf{gráficos y productos característicos} en mm-wave (p.,ej., diagramas tiempo--frecuencia y diagnósticos de polarización).
\end{enumerate}
\end{itemize}

\medskip
En esta memoria, no se contempla reentrenar modelos ni desarrollar \emph{backends} instrumentales; el foco es \textbf{inferencias y orquestación} a partir de modelos existentes y datos crudos/reducidos.

\medskip
\noindent\textbf{Resultados esperados (medibles).} (i) \emph{Pipeline} E2E ejecutable por línea de comandos y/o servicio, con documentación y pruebas; (ii) reporte comparativo de desempeño (\emph{recall}, \emph{precision}, \emph{throughput}, latencia por GB/archivo); (iii) conjunto de figuras/\emph{notebooks} ilustrativos para régimen de alta frecuencia.

\begin{figure}[h]
\centering
% \includegraphics[width=0.9\textwidth]{workflow_propuesto.pdf}
\caption{\textbf{Sugerencia de figura:} Flujo actual (dedispersión + candidatos + curaduría) versus flujo propuesto (detección DL + clasificación DL + reporte). Señalar puntos de latencia y reducción de falsos positivos.}
\end{figure}

\begin{table}[h]
\centering
\caption{\textbf{Sugerencia de tabla}: diferencias prácticas entre bandas decimétricas y mm-wave relevantes para el diseño del pipeline.}
\begin{tabular}{p{0.30\textwidth} p{0.30\textwidth} p{0.30\textwidth}}
\toprule
\textbf{Aspecto} & \textbf{0.3--3\,GHz} & \textbf{30--100\,GHz} \\
\midrule
Retardo por dispersión & Alto; patrones claros en tiempo--DM & Bajo; patrones atenuados \\
RFI/sistemáticas & RFI ancha banda & Atmósfera, estabilidad de fase, distinta RFI \\
Productos útiles & Waterfalls, tiempo--DM & Tiempo--frecuencia, polarización/diagnósticos alternativos \\
Disponibilidad de software & Pipelines consolidados & Prototipos, sin estándar general \\
\bottomrule
\end{tabular}
\end{table}

\medskip
\noindent\textbf{Nota.} Este capítulo define el \emph{problema} (no la solución detallada). La solución propuesta se desarrolla en capítulos posteriores (diseño del \emph{pipeline}, validación y resultados), siguiendo la lógica de una memoria profesional.\footnote{Las referencias completas a DRAFTS, herramientas clásicas y trabajos en alta frecuencia se incorporarán en el capítulo de marco conceptual y bibliografía.}