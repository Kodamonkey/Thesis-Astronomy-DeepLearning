\secnumbersection{ESTADO DEL ARTE}

\section{Pipelines y herramientas de búsqueda de pulsos}
Se revisan suites como PRESTO y Heimdall, y otras herramientas relevantes para búsquedas de pulsos individuales, destacando fortalezas y limitaciones en escenarios operativos.

\textbf{TODO}: Añadir referencias y una tabla breve comparando PRESTO vs Heimdall vs otros (entradas, desempeño, salida, facilidad de integración).

\section{DRAFTS: detector CenterNet y clasificador ResNet}
Se describe el enfoque DRAFTS, su arquitectura de detección en tiempo--DM y la clasificación binaria, explicitando qué resuelve y qué queda fuera del alcance (pipeline operativo extremo a extremo).

\textbf{TODO}: Resumir hiperparámetros y tamaños de entrada de los modelos de DRAFTS empleados.

\section{Estudios a frecuencias altas}
Se resumen intentos y hallazgos en bandas altas, identificando brechas y desafíos abiertos.

\textbf{TODO}: Incorporar 1–2 trabajos recientes en mm-wave y su pertinencia para ALMA.

\section{Síntesis del gap abordado}
Se articula el vacío entre modelos de ML y su operacionalización, y la extensión a mm-wave propuesta en este trabajo.

\section{Arquitecturas operativas: CHIME/FRB, ASKAP/CRAFT, MeerTRAP}
Se reseñan arquitecturas en tiempo real/\textit{near-real-time}, mecanismos de backend y publicación de candidatos, como referencia para requerimientos y comparación.

\textbf{TODO}: Esbozar 4–5 bullets por arquitectura con su pipeline y latencias.

\section{Mitigación de RFI}
Panorama de estrategias aceptadas (AOFlagger, enfoques morfológicos/estadísticos) y su relación con el preprocesado del pipeline.


