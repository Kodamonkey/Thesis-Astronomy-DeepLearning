\secnumbersection{MARCO TEÓRICO}

\section{FRBs y pulsos individuales: DM, dispersión, ancho y S/N}
Las FRBs son pulsos de duración típica de milisegundos, caracterizados por su \textit{Medida de Dispersión} (DM), ancho intrínseco/observado y relación señal/ruido (S/N). La dispersión en el medio ionizado introduce un retardo en frecuencia que se aproxima por
\[\Delta t\,\,\mathrm{[ms]} \approx 4.15\times10^3\,\mathrm{DM}\,\left(\nu_1^{-2}-\nu_2^{-2}\right),\]
donde \(\nu\) se expresa en MHz. El ensanchamiento por dispersión y efectos instrumentales (\textit{smearing}) limita la sensibilidad si excede el ancho del pulso.

\section{Plano tiempo--DM y patrón ``bow-tie''}
La búsqueda de pulsos individuales se realiza en el plano tiempo--DM, donde una ráfaga astrofísica genera un patrón en ``corbatín'' (\textit{bow-tie}). La apertura del patrón depende de la cobertura en frecuencia y de la DM del evento; el máximo S/N aparece cerca de la DM verdadera.

\section{RFI: criterios de discriminación astrofísico vs terrestre}
La RFI suele concentrarse en bandas estrechas, presenta DM cercana a cero y carece de la dispersión cromática propia de señales cósmicas. Criterios como consistencia en sub-bandas, simetría temporal, estabilidad de DM y persistencia espacial ayudan a discriminar eventos astrofísicos.

\section{Dedispersión y waterfalls (sin y con dedispersión)}
Los \textit{waterfalls} crudos muestran la dispersión en frecuencia; tras dedispersar a una DM candidata, un pulso real se alinea temporalmente a través de canales. La inspección conjunta tiempo--DM y \textit{waterfalls} dedispersados es clave para validar candidatos.

\section{Particularidades a alta frecuencia (mm-wave)}
A frecuencias milimétricas el retardo por dispersión es menor, atenuando el ``bow-tie'' y reduciendo la ganancia de S/N asociada a la dedispersión. Esto exige ampliar el rango y el paso de DM o estrategias de validación por sub-bandas y clasificación complementaria.

\section{Telescopios y datos relevantes}
Se consideran instrumentos y formatos representativos: FAST y Effelsberg (banda L), y ALMA en modo \textit{phased} para frecuencias milimétricas. Formatos habituales: PSRFITS/SEARCH y \texttt{FIL}, con metadatos esenciales como \texttt{TBIN}, \texttt{NCHAN}, ancho de banda, frecuencias y tiempos de inicio en MJD.

\section{Convenciones de polarización (PSR/IEEE) y PSRFITS/PSRCHIVE}
Se adopta la convención PSR/IEEE para Stokes \(I,Q,U,V\). La correcta interpretación y almacenamiento de polarización en PSRFITS y su tratamiento en PSRCHIVE son relevantes para análisis avanzados y mitigación de RFI.

\section{Trazabilidad temporal}
Se resguarda trazabilidad mediante sellos de tiempo absolutos (MJD topocéntrico) y consistencia con modelos temporales (p.ej., TEMPO2), asegurando coherencia entre bloques y reproductibilidad de ventanas temporales.
